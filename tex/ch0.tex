\setcounter{chapter}{-1}
\chapter{Introduction}

\emph{Mathematical Components}
is the name of a library of formalized mathematics for the
\Coq{} system.  It covers a variety of
topics, from the theory of basic data structures (e.g., numbers, lists, finite
sets) to advanced results in various flavors of algebra. This library
constitutes the infrastructure for the machine-checked proofs of the
Four Color Theorem~\cite{Gonthier08} and of the
Odd Order Theorem~\cite{gonthier:hal-00816699}.

The reason of existence of this book is to break down the barriers to entry.
While there are several books around covering the usage of the
\Coq{} system~\cite{BC04,SF,CPDT}
and the theory it is based
on~\cite[chapter 4]{Coq:manual}\cite{paulinmohring:hal-01094195,hottbook},
the \mcbMC{} library is built
in an unconventional way.  As a consequence, this book provides a
non-standard presentation of \Coq{}, putting upfront the formalization
choices and the proof style that are the pillars of the library.

This books targets two classes of public.  On one hand, newcomers,
even the more mathematically inclined ones, %may
find a soft
introduction to the programming
language of \Coq{}, Gallina, and the Ssreflect proof language.
On the other hand accustomed \Coq{} users %will hopefully
find a
substantial account of the formalization style that made the \mcbMC{}
library possible.


By no means does this book pretend to be a complete description of \Coq{} or
Ssreflect: both tools already come with a comprehensive user
manual~\cite{Coq:manual,ssrman}.
In the course of the book, the reader is nevertheless invited to
experiment with  a large library of formalized concepts and she is
given as soon as possible sufficient tools to prove non-trivial
mathematical results by reusing parts of the library. By the end of
the first part, the reader has learnt how to prove formally the
infinitude of prime numbers, or the correctness of the Euclidean's
division algorithm, in a few lines of proof text.

\paragraph{Acknowledgments.} The authors wish to thank
Reynald Affeldt, Guillaume Allais, Sophie Ber\-nard, Alain Giorgetti,
Laurence Rideau, Damien Rouhling, Michael Soegtrop for their comments
on earlier versions of this text. They are specially grateful to
Pierre Jouvelot, Darij Grinberg and Anton Turnov for their careful
proofreading and for their suggestions. Many thanks to Hanna for the
illustrations.

%%%%%%%%%%%%%%%%%%%%%%%%%%%%%%%%%%%%%
\section*{Structure of the book}

% The book is divided into three parts.
% The first two parts of the book are conceived to be read linearly,
% while the third one can be read in any order.

In its current state, the book has two parts which are meant to be
read in order. Some more advanced sections are identified with stars,
with more stars for more advanced topics.

\subsection*{Part 1: Definitions and proofs}

This part introduces two languages and a formalization approach.

The first language is \emph{Gallina}, the mathematical foundation of the \Coq{}
system:  For the expert reader, the type theory called \mcbCIC{}.  Such
language is both a programming and a specification language: it is used to
declare data, define programs, specify their behavior and represent their
correctness proofs.  Its most intriguing features are two: a rich system of
types, that can depend on other types or even values, and the notion of
computation, that is considered a first-class citizen, i.e., a true form of
proof.

The second language, \mcbSSR{} (\emph{Ssreflect} for short), is used to write
proofs.  Its most characterizing features are two: it lets one write
concise and stable proof scripts enabling the development of large
libraries over a long time frame, and provides good support for the
formalization approach the \mcbMC{} library is based on, from which
its name is also taken.

The \emph{\mcbSSR{}} approach was initially conceived for the formal proof
of the Four Colors Theorem, and it later served as the pillar for
the \mcbMC{} library and the formal proof of the Odd Order Theorem.
Such approach takes major advantage from the symbolic computation
capabilities provided by Gallina.

The \Coq{} system provides a rich platform in which all this is
put together.  \Coq{} implements the Gallina language, and a checker for it.
It also provides a platform for the Ssreflect proof language and
many other facilities that are crucial for the \mcbMC{} library, like
type inference.

\subsection*{Part 2: Formalization techniques}

This part provides the tools to build a large library of formalized
mathematics.  In particular it presents a powerful form of automation
and a formalization technique that makes it possible to organize
concepts in a rational way and easily define new ones by linking them
to the already existing ones.

Automation is provided by \emph{programming type inference}.
The \Coq{} system provides a user-extensible type inference
algorithm.  It can be extended with declarative programs
giving canonical solutions to otherwise unsolvable problems.
Such solutions typically involve notions and theorems that
are part of the \mcbMC{} library.  By programming type inference
one can hence teach \Coq{} the contents of the library.  The system
is then able to reconstruct non-trivial missing piece of information,
as an informed reader typically does when reading a mathematical text.

Formalized knowledge is organized by means of interfaces (in the spirit of
algebraic structures) and relations between them.  Type inference is programmed
to play the role of a librarian and recognize when an abstract theory has the
right to apply to a specific example.

Finally the rich language of \Coq{} lets one define new concepts
by refining existing ones, typically by gluing an object with
a proof of some extra property.  Type inference is programmed
to transport all the theory available on
the original concept to the derived one.

% \subsection*{Part 3: Mathematics in Mathematical Components}

% This part provides a, unfinished by design, panoramic view of the
% \mcbMC{} library.  It interleaves a catalog of formalized theories with
% formalization choices that play a crucial role in the
% definition of the formal object.

%%%%%%%%%%%%%%%%%%%%%%%%%%%%%%%%%%%%%%%%%%%%%%%%%%%%%
\section*{How to use the book}

\subsection*{Conventions}
Sections that are labeled with one $(\star)$ are of medium complexity and are
of interest to the reader willing to extend the \mcbMC{} library.  Sections
that are labeled with two $(\star\star)$ are of high complexity and are of
interest to the reader willing to understand the technical implementation
details of the \mcbMC{} library.

Tricky details typically overlooked by beginners are signaled as follows:
\gotcha{Mind this detail.}

Advice one should keep in mind are signaled as follows:
\mantra{Remember this advice.}

\Coq{} code is in \C{typewriter} font and \C{(surrounded by parentheses)}
when it occurs the middle of the text.  Longer snippets are in boxes with line
numbers like the following one:

\begin{coq}{}{title=A sample snippet}
Example Gauss n : \sum_(0 <= i < n.+1) i = (n * n.+1) %/ 2.
Proof.
elim: n =>[|n IHn]; first by apply: big_nat1.
rewrite big_nat_recr //= IHn addnC -divnMDl //.
by rewrite mulnS muln1 -addnA -mulSn -mulnS.
Qed.
\end{coq}

Code snippets are often accompanied by the goal status
printed by \Coq{}.

\begin{coqout}{}{title=Output from \Coq{} after line 3}
n : nat
IHn : \sum_(0 <= i < n.+1) i = (n * n.+1) %/ 2
============================
\sum_(0 <= i < n.+2) i = (n.+1 * n.+2) %/ 2
\end{coqout}

Names of components one can \C{Require} in \Coq{} are written
like \lib{ssreflect} or \lib{fintype}.

\subsection*{Running examples in the \Coq{} system}
The contents of this book is mostly about interacting with a computer
program consisting of the \Coq{} system and the \mcbMC{} library.  Many
examples are given, and we advise readers to experiment with this program,
after having installed the \Coq{} system and the \mcbMC{} library on a
computer.  Documentation on how to install \Coq{} is available
at \url{http://coq.inria.fr}, while documentation on how to install
the \mcbMC{} library is available at
\url{https://math-comp.github.io/math-comp/}.

The initial revision of the book was written and tested against
version 8.5 of the \Coq{} system and version 1.6 of the \mcbMC{} library.

There are a variety of ways to run the \Coq{} system: a command line is
provided under the name \texttt{coqtop}, while a windowed interface is
provided under the name \texttt{coqide}.  The \Coq{} community also develops
alternative approaches to integrate \Coq{} in their preferred programming
environment.  For instance, there exist special extensions of \Coq{} for the
popular \texttt{emacs} programming editor (known as \texttt{proof general})
and for the \texttt{Eclipse} programming environment
(known as \texttt{coqoon}).  These extensions and similar projects can easily
be found by a search on the internet.

When starting a \Coq{} session, a few commands must be sent to the \Coq{}
system to tell it to load the \mcbMC{} library and to configure its behavior
so it matches the usual programming style of the \mcbMC{} developers:

\begin{coq}{}{title={Default starting header}}
From mathcomp Require Import all_ssreflect.
Set Implicit Arguments.
Unset Strict Implicit.
Unset Printing Implicit Defensive.
\end{coq}
The first line actually instructs the \Coq{} system to load the first level
of the \mcbMC{} library.  More advanced levels are available under names
\lib{all\_fingroup}, \lib{all\_algebra}, \lib{all\_solvable}, \lib{all\_field},
and \lib{all\_character}.  While the first chapters of this book
rely mainly on the first
level, later chapters will rely on the other levels.  This will be clearly
stated as the topics evolve.
